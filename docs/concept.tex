\documentclass[8pt,a4paper]{article}
\usepackage[utf8]{inputenc}
\usepackage[T1]{fontenc}
\usepackage{isodate}
\usepackage[polish]{babel}
\usepackage{enumerate}
\usepackage{makecell}
\usepackage{indentfirst}
\setlength{\parskip}{0.2em}

\usepackage{xcolor}
\usepackage{listings}

\definecolor{commentColour}{rgb}{0,0.6,0}
\definecolor{mGray}{rgb}{0.5,0.5,0.5}
\definecolor{mPurple}{rgb}{0.58,0,0.82}
\definecolor{backgroundColour}{rgb}{0.98, 0.98, 0.98}

\lstdefinestyle{CStyle}{
    backgroundcolor=\color{backgroundColour},   
    commentstyle=\color{commentColour},
    keywordstyle=\color{blue},
    numberstyle=\tiny\color{mGray},
    stringstyle=\color{mPurple},
    basicstyle=\footnotesize,
    breakatwhitespace=false,         
    breaklines=true,                 
    captionpos=b,                    
    keepspaces=false,                 
    numbers=left,                    
    numbersep=0pt,                  
    showspaces=false,                
    showstringspaces=false,
    showtabs=false,                  
    tabsize=2,
    language=C
}

\title{SOI - Semafory \\ \large Koncepcja}
\author{Jakub Mazurkiewicz}

\begin{document}

\maketitle

\section{Procesy}
W systemie zostaną uruchomione cztery procesy - dwóch producentów i dwóch konsumentów. Zadaniem producentów będzie wytworzenie danych typu \texttt{int} i umieszczenie ich na zmianę w jednej z dwóch kolejek FIFO. Konsumenci (\texttt{process-c} i \texttt{process-d}) będą wyjmować dane z przypisanej im kolejki, po czym wyświetlą informację o tym, od którego producenta pochodzi dana oraz czas odbioru. Dostawcę liczby będziemy rozpoznawać na podstawie podzielności przez 3:
\begin{itemize}
    \item Producent 1 (\texttt{process-a}) będzie produkował losowe liczby podzielne przez 3
    \item Producent 2 (\texttt{process-b}) będzie produkował dwie kolejne liczby niepodzielne przez 3 (czyli $1, 2, 4, 5, ...$)
\end{itemize}

\section{Narzędzia}
Kolejka FIFO zostanie zrealizowana jako struktura posiadająca wskaźnik do pamięci współdzielonej (\texttt{shm}) zawierającej przesyłane dane oraz dodatkowe atrybuty.

\begin{lstlisting}[style=CStyle]
#define MAX_SIZE 20
struct main_queue_t {
    int* memory;
    // ewentualne dodatkowe parametry
};
\end{lstlisting}

\noindent Dla kolejek powstaną również semafory chroniące sekcję krytyczną i informujące o ilości dostępnych miejsc w kolejce.

\noindent Podsumowanie:
\begin{table}[h!]
  \begin{center}
    \begin{tabular}{c|c}
      \textbf{Narzędzie} & \textbf{Wybór} \\
      \hline
      System & Ubuntu 20.04/18.04 \\
      Umiejscowienie kolejki FIFO & Pamięć współdzielona (\texttt{<sys/shm.h>}) \\
      Synchronizacja & Semafory (\texttt{<semaphore.h>}) \\
    \end{tabular}
  \end{center}
\end{table}

\end{document}
